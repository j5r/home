

%%%%%%%%%%%%%%%%%%%%%%%%%%%%%%%%%
\chapter{Modelos de cadeias de Markov controlados}
%%%%%%%%%%%%%%%%%%%%%%%%%%%%%%%%%
This chapter presents basic results for stochastic systems rra.odeled as finite state controlled Markov chains. In the case of complete observations and feedback laws depending only on the current state", the state process is a Markov chain. Asymptotic properties of Markov chains are reviewed. Infinite state Markov chains are studied briefly. Finally, a
technique is presented that reduces the study of certain contiIl.uous time Markov processes to discrete time Markov chains. The techniq ue is illustrated by an example of a queuing system.




Este capítulo coleciona resultados básicos para modelos de sistemas estocásticos lineares com ruídos nas entradas e erros de mensuração. É dada atenção especial ao caso em que as perturbações são Gaussianas. Equações recursivas são derivadas para a covariância e a média dos processos do estado e das Observações. As suas propriedades assintóticas são relacionadas à estabilidade das equações de estado. Finalmente, os processos de estado de sistemas lineares Gaussianos são relacionados a processos Gauss-Markovianos.




%%%%%%%%%%%%%% s
\section{Sistemas Lineares Gaussianos}
%%%%%%%%%%%%%% s
Para uma variável aleatória $n$-dimensional $z$, escrevemos
\begin{equation}\label{3.1.1}
	z \sim N(\bar z, \Sigma) \quad
	\text{ ou }\quad p(z) \sim N(\bar z,
	\Sigma)
\end{equation}
para indicar que $z$ é Gaussiana ou normal com média $\bar z$ e covariância $\Sigma$. Lembre que a PD\nomenclature{PD}{Probability Distribution --- distribuição de probabilidade.} de $z$ é dada por
\begin{equation}\label{3.1.2}
	p(z)=\Big[(2\pi)^n \det(\Sigma)
	\Big]^{-1/2}\exp\left(-1/2\;
	(z-\bar z)^T\Sigma^{-1}(z-\bar z)\right).
\end{equation}

Considere o {\bf sistema estocástico linear}
\begin{subequations}\label{3.1.3}
	\begin{align}
	 x_{k+1}= &\; Ax_k+Bu_k+Gw_k,
	 \label{3.1.3a}\\
	y_k =&\; Cx_k + Hv_k,
	\label{3.1.3b}
	\end{align}
\end{subequations}
onde $x_k,\;u_k,\;y_k,\;w_k$ e $v_k$ são reais de dimensões $n,\;m,\;p,\;g$ e $h$ respectivamente, e as matrizes $A,\;B,\;G,\;C$
e $H$ são constantes de dimensões apropriadas.
As variáveis aleatórias básicas independentes $x_0,\; w_0,\; v_0,\; w_1,\;v_1,\dots$ são assumidas Gaussianas:
\begin{equation}\label{3.1.4}
	x_0\sim N(\bar x_0,\Sigma_0),
	\quad w_k\sim N(0,Q),
	\quad y_k\sim N(0,R).
\end{equation}

Calculamos a probabilidade de transição (em um passo) para esse sistema. Para isso, podemos diretamente utilizar a Equação \eqref{2.6.7}, mas é mais instrutivo partirmos dos princípios básicos. Como $w_k$ é Gaussiana, $Gw_k$ também o é. Sua média é $\E Gw_k=G\E w_k=0$. Sua covariância é $\E (Gw_k)(Gw_k)^T=G\E w_kw_k^TG^T=GQG^T$ e então $Gw_k\sim N(0,GQG^T)$.

Por hipótese, as VAs básicas são independentes. Portanto, para qualquer lei de realimentação $g$, $w_k$ é independente de $x_k$ e de $u_k$. Particularmente, vemos de \eqref{3.1.3a} que a distribuição condicional de $x_{k+1}$ dado $x_k$ e $u_k$ é a mesma que a PD de $Gw_k$ com a média transladada por $Ax_k+Bu_k$. Assim, a probabilidade de transição do estado é
\begin{equation}\label{3.1.5}
	p(x_{k+1}\;|\;x_k,u_k) \sim
	N(Ax_k+Bu_k,GQG^T).
\end{equation}
Substituindo em \eqref{3.1.2} nos dá explicitamente
\[
\begin{split}
	p(x_{k+1}\;|\;x_k,u_k) &= \Big[(2\pi)^n
	\det(GQG^T)
	\Big]^{-1/2}\\&\exp\left(-1/2\;
	(x_{k+1}-Ax_k-Bu_k)^T  (GQG^T)^{-1}
	(x_{k+1}-Ax_k-Bu_k)
	\right).
\end{split}
\]

De forma análoga mostramos que
\begin{equation}\label{3.1.6}
	p(y_k\;|\; x_k) \sim N(Cx_k,HRH^T).
\end{equation}

A seguir, calculamos a probabilidade de transição em $m$ passos. Seja $g$ qualquer lei de realimentação na qual
$g_{k+1}\equiv u_{k+1},\;\dots,\; g_{k+m-1}\equiv u_{k+m-1}$ sejam funções constantes. Considere também os processos de estado e saída $\{x_k\}$ e $\{y_k\}$ respectivamente. Pela Definição \ref{2.6.14}, a probabilidade de transição em $m$ passos é a distribuição condicional de $x_{k+m}$ dados $x_k,\;u_k,\;\dots,\;u_{k+m-1}$ e não depende de $g$. Em vez de usarmos  \eqref{2.6.14}, é mais fácil calcular essa distribuição condicional diretamente. De \eqref{3.1.3a}
\begin{equation}\label{3.1.7}
	x_{k+m}=A^mx_k+\sum_{j=0}^{m-1}
	A^{m-1-j}Bu_{k+j}+
	A^{m-1-j}Gw_{k+j}.
\end{equation}
As VAs $\{w_{k+j},\;j\ge0\}$ são Gaussianas e independentes de $x_k$ e $u_k$, assim como $u_{k+j},\;j\ge1$, já que estas são constantes. Portanto, de \eqref{3.1.7},
\begin{equation}\label{3.1.8}
	p(x_{k+m}\;|\;x_k,\;u_k,\;\dots,\;u_{k+m-1})
	\sim N\left(A^mx_k+\sum_{j=0}^{m-1}
	A^{m-1-j}Bu_{k+j},\;\;
	\Sigma_{k+m\;|\;k}\right)
\end{equation}
em que $\Sigma_{k+m\;|\;k}:=\cov\left(\sum_{j=0}^{m-1}A^{m-1-j}Gw_{k+j}\right)$. Essa covariância pode ser facilmente avaliada usando a independência entre os $w_k$:
\begin{equation}\label{3.1.9}
\begin{aligned}
	\Sigma_{k+m\;|\;k}:=&
	\cov\left(\sum_{j=0}^{m-2}A^{m-1-j}
	Gw_{k+j}\right)+\cov(Gw_{k+m-1})\\
	=&\cov\left(A\sum_{j=0}^{m-2}A^{m-2-j}
	Gw_{k+j}\right)+GQG^T\\
	=&A\cov\left(\sum_{j=0}^{m-2}A^{m-2-j}
	Gw_{k+j}\right)A^T+GQG^T\\
	=& A\;\Sigma_{k+m-1\;|\;k}\;A^T+GQG^T.
\end{aligned}
\end{equation}
Esta equação linear recursiva pode ser resolvida para $\Sigma_{k+m\;|\;k}$, $m>0$, começando com
\begin{equation}\label{3.1.10}
\Sigma_{k\;|\;k}=0.
\end{equation}
De \eqref{3.1.9} e \eqref{3.1.10}, vemos que $\Sigma_{k+m\;|\;k}$ depende apenas de $m$ e não de $k$.

Note que na discussão acima, as funções de realimentação $g_1,\;\dots,\;g_k$ (em contraste com $g_i,\;i\ge k$) não precisariam ser funções constantes. De fato, $g_i$, com $i\le k$, poderia ser qualquer função não linear das informações disponíveis $\{y^i\}$. Se elas fossem não lineares, então o processo de controle não seria Gaussiano, e então o processo dos estados (que é determinado pelas probabilidades \emph{não condicionais}) também não o seriam. Mesmo assim, como vimos, as probabilidades de transição (que forem \emph{condicionais}) são sim Gaussianas. Resumimos este resultado num lema.
\begin{Lema}
A probabilidade de transição em $m$ passos para o sistema \eqref{3.1.3} e \eqref{3.1.4} com qualquer lei de realimentação $g_{k+1}\equiv u_{k+1},\;\dots,\; g_{k+m-1}\equiv u_{k+m-1}$, é dada pela densidade Gaussiana
\[
p(x_{k+m}\;|\;x_k,\;u_k,\;\dots,\;u_{k+m-1})
	\sim N\left(A^mx_k+\sum_{j=0}^{m-1}
	A^{m-1-j}Bu_{k+j},\;\;
	\Sigma_{k+m\;|\;k}\right).
\]
Além disso, a matriz de covariância $\Sigma_{k+m\;|\;k}$ pode ser calculada pela equação linear de diferenças \eqref{3.1.9} e \eqref{3.1.10}.
\end{Lema}

Suponha agora, que a lei de controle é de malha aberta o tempo todo, isto é, $g_i\equiv u_i$ para todo $i\ge0$ (em particular, também para $i\le k$). Então o processo de estado $\{x_k\}$ é Gaussiano e então sua PD (não condicional) é completamente determinada pelas suas médias (não condicionais) $\E x^g_k$ e a função de covariância (não condicional) $\Sigma_{k+m,\;k}:=\cov(x_{k+m},x_k)$.

Para obter $p^g_{x_k}$, a qual determina $\E x^g_k$ e $\Sigma_{k,\;k}:=\cov(x_{k},x_k)=:\Sigma_k$, perceba que
\[
x_{k}=A^kx_0+\sum_{j=0}^{k-1}A^{k-1-j}Bu_{j}+A^{k-1-j}Gw_{j}.
\]
Os três termos do membro direito são todos independentes e Gaussianos. Consequentemente,
\[
p^g_{x_k}\sim N(\bar x_k, \Sigma_k),
\]
em que
\begin{align}
	\bar x_k &=A^k\bar x_0+
	\sum_{j=0}^{k-1}A^{k-1-j}Bu_{j}\nonumber\\
	\Sigma_k&=\cov\left(A^kx_0+
	\sum_{j=0}^{k-1}A^{k-1-j}Gw_{j}\right)
	\nonumber\\
	&=\cov\left(A\left(A^{k-1}x_0+
	\sum_{j=0}^{k-2}A^{k-2-j}Gw_{j}\right)
	\right)
	+\cov(Gw_{k-1})
	\nonumber\\
	&=A \Sigma_{k-1}A^T+GQG^T, \quad k\ge1,
	\label{3.1.12}\\
	\Sigma_0&=\cov(x_0)\label{3.1.13}.
\end{align}
Assim, a PD de $\{x_k\}$ pode ser calculada recursivamente.

Para obter a função de covariância $\Sigma_{k+m,\;k}$, lembre que
\[
	\Sigma_{k+m,\;k}:=\E(x_{k+m}
	-\bar x_{k+m})(x_k-\bar x_k)^T.
\]
De \eqref{3.1.7},
\[
	x_{k+m}-\bar x_{k+m}=A^m(x_{k}
	-\bar x_{k})+\sum_{j=0}^{m-1}
	A^{m-1-j}Gw_{k+j},
\]
e, sendo $x_k$ independente de $w_{k+j}$, para $j\ge0$,
\begin{equation}\label{3.1.14}
	\Sigma_{k+m,\;k}=A^m\Sigma_{k}.
\end{equation}
Note que a média $\bar x_k$ depende do processo de controle $\{u_k\}$, mas a covariância $\Sigma_{k+m,\;k}$ não.

Quando $g$ é de malha aberta, o processo $\{y_k\}$ também é Gaussiano. De \eqref{3.1.3b}, vemos que
\[
	p^g_{y_k} \sim N(\bar y_k, \Sigma^y_k),
\]
onde a média é
\[
	\bar y_k:=\E y_k=C\E x_k
	=C\left(A^k\bar x_0+
	\sum_{j=0}^{k-1}A^{k-1-j}Bu_{j}\right),
\]
e a covariância é
\begin{equation}\label{3.1.15}
	\Sigma^y_k=\cov(Cx_k)+\cov(Hv_k)
	=C\Sigma_k C^T+HRH^T.
\end{equation}
A covariância de $y_{k+m}$ e $y_k$ é
\[
	\Sigma^y_{k+m,\;k}:=\E (y_{k+m}
	-\bar y_{k+m})(y_k-\bar y_k)^T.
\]
De \eqref{3.1.3b} e \eqref{3.1.14}, e usando a independência de $\{x_k\}$ e $\{v_k\}$,
\begin{align}
	\Sigma^y_{k+m,\;k} &= C\Sigma_{k+m,\;k}
	C^T,\quad m\ge 1,
	\nonumber\\
	&=CA^m\Sigma_{k}C^T,\quad m\ge 1.
\end{align}
O próximo exercício examina o sistema estocástico linear variante no tempo.
\begin{Exercicio}\label{3.1.17}
Suponha que em vez do sistema invariante no tempo \eqref{3.1.3} -- \eqref{3.1.4}, tenhamos o sistema variante no tempo
\begin{align*}
	x_{k+1} &=A_k x_k+ B_k u_k+ G_k w_k,\\
	y_k &= C_k x_k + H_k v_k,
\end{align*}
e que as variáveis básicas sejam independentes e Gaussianas:
\[
	x_0 \sim N(\bar x_0, \Sigma_0), \quad
	w_k \sim N(0,Q_k), \quad
	v_k \sim N(0,R_k).
\]
Mostre que a probabilidade de transição em $m$ passos é Gaussiana,
\[
	p(x_{k+m}\;|\;x_k,\,u_k,\,\dots,u_{k+m-1})
	\sim N\left(A_{k+m-1}\dots A_kx_k+
	\sum_{j=0}^{m-1}A_{k+m-1}\dots A_{k+j+1}
	B_{k+j}u_{k+j},\;\; \Sigma_{k+m\;|\;k}
	\right).
\]
Além disso, a covariância pode ser calculada pela equação linear de diferenças variantes no tempo
\begin{align*}
	\Sigma_{k+m\;|\;k} &=
	A_{k+m-1}\,\Sigma_{k+m-1\;|\;k}\,A_{k+m-1}^T
	+G_{k+m-1}Q_{k+m-1}G_{k+m-1}^T,\\
	\Sigma_{k\;|\;k} &= 0.
\end{align*}
\end{Exercicio}

Ao longo do exposto, assumimos que o ruído $\{w_k\}$ que entra no sistema é um processo independente. O próximo exercício mostra como estender essa suposição ampliando o vetor de estado.

\begin{Exercicio}\label{3.1.18}
Suponha que a sequência de ruídos $\{w_0,\,w_1,\,\dots\}$ não seja independente, de modo que o sistema \eqref{3.1.3}--\eqref{3.1.4} não é um sistema estocástico linear como definido. No entanto, suponha que o próprio $\{w_r\}$ seja a saída de um sistema linear
\begin{align*}
	\xi_{k+1} &= F \xi_k+\epsilon_k,\\
	w_k&=D \xi_k+\delta_k,
\end{align*}
em que $\{x_0,\xi_0,\dots,v_0,\dots,\epsilon_0,\dots,\delta_0,\dots\}$ sejam Gaussianas independentes. Mostre que $\{y_k\}$ pode ser escrito como a saída de um sistema estocástico linear.

[Dica: considere o vetor de estados $\varsigma_k:=(x_k^T,\xi_k^T)^T$.]
\end{Exercicio}

Assim nossa definição de sistemas lineares estocásticos engloba os casos nos quais a entrada de ruído seja  gerada por uma filtragem de ruído branco Gaussiano via algum sistema linear estocástico.





%%%%%%%%%%%%%% s
\section{Sistemas Lineares não Gaussianos}
%%%%%%%%%%%%%% s
Considere o sistema linear \eqref{3.1.3}, mas suponha que as variáveis básicas independentes $x_0,w_0,\dots,v_0,\dots$ não sejam Gaussianas. Mas assuma que suas médias e covariâncias sejam as mesmas de antes:
\[
\begin{aligned}
	\E x_0&=\bar x_0, \quad \E w_k = 0,
	\quad \E v_k=0,\\
	\cov(x_0)&=\Sigma_0, \quad \cov(w_k)=W,
	\quad \cov(v_k)=R.
\end{aligned}
\]
Considere um controle de malha aberta de modo que $g_k=u_k,\;k\ge0$, sejam funções constantes.

Sejam
\[
\begin{aligned}
	\bar x_{k+m\;|\;k}&:= \E(x_{k+m}
	\;|\;x_k,u_k,\dots,u_{k+m-1}),\\
	\Sigma_{k+m\;|\;k}&:=
	\E\left([x_{k+m}-\bar x_{k+m\;|\;k}]
	[x_{k+m}-\bar x_{k+m\;|\;k}]^T
	\;\big|\;x_k,u_k,\dots,u_{k+m-1}\right),
\end{aligned}
\]
a média e covariância condicionais de $x_{k+m}$ dado $x_k$ respectivamente.

De \eqref{3.1.7} segue que
\[
\begin{aligned}
	\bar x_{k+m\;|\;k}&= A^mx_k+
	\sum_{j=0}^{m-1}A^{m-1-j}Bu_{k+j},\\
	\Sigma_{k+m\;|\;k} &=
	\cov\left(\sum_{j=0}^{m-1}A^{m-1-j}
	Gw_{k+j}\right)\\
	&=A\,\Sigma_{k+m-1\;|\;k}\,A^T+GQG^T.
\end{aligned}
\]
Comparando com \eqref{3.1.8}, vemos que os dois primeiros momentos condicionais são os mesmos que no caso Gaussiano. No entando, a distribuição condicional não precisa ser Gaussiana.

De modo semelhante, a média e covariância (não condicionais)
\[
	\bar x_k:=\E x_k,\quad
	\Sigma_k:=\E(x_k-\bar x_k)(x_k-\bar x_k)^T
\]
são dadas também pela \eqref{3.1.12}, quando o controle é de malha aberta.

Essa coincidência para os dois primeiros momentos também valem para o caso variante no tempo do Exercício \ref{3.1.17}.





%%%%%%%%%%%%%% s
\section{Propriedades assintóticas}
%%%%%%%%%%%%%% s
Estudaremos o comportamento das matrizes de covariância $\Sigma_k$ dadas por \eqref{3.1.12}--\eqref{3.1.13} conforme $k\to\infty$, as quais reproduzimos aqui:
\begin{align}
	\Sigma_k &= A\Sigma_{k-1}A^T+GQG^T,
	\quad k\ge 1, \label{3.3.1}\\
	\Sigma_0&=\cov(x_0).
\end{align}
Substituições recursivas leva a
\begin{equation}\label{3.3.3}
	\Sigma_k = A^k\Sigma_0(A^k)^T+
	\sum_{j=0}^{k-1}A^jGQG^T(A^j)^T.
\end{equation}
Para obter a convergência de $\Sigma_k$, precisamos que a série seja somável, e para isso precisamos que a matriz $A^j$ convirja para $0$.

Lembre que a matriz $A$ é dita \textbf{estável} se seus autovalores são estritamente menores que $1$ em valor absoluto.

\begin{Teo}\label{3.3.4}
Suponha que $A$ seja estável. Então existe uma matriz positiva semidefinida $\Sigma_\infty$ tal que $\lim_{k\to \infty}\Sigma_k=\Sigma_\infty$. Além disso, $\Sigma_\infty$ é a única solução da equação
\begin{equation}\label{3.3.5}
	\Sigma_\infty = A \Sigma_\infty A^T
	+GQG^T.
\end{equation}
\end{Teo}

\begin{Dem}
Sendo $A$ uma matriz estável, existem números $K<\infty$ e $0<\alpha<1$ de modo que todas as entradas da matriz $A^j$ são menores que $K\alpha^j$ em valor absoluto. Com isso, o primeiro termo em \eqref{3.3.3} desaparece e a soma converge conforme $k\to\infty$, e então
\[
	\lim_{k\to\infty}\Sigma_k=
	\Sigma_\infty:=
	\sum_{j=0}^{\infty}A^jGQG^T(A^j)^T.
\]
Depois, como $\Sigma_k$ converge para $\Sigma_\infty$, podemos tomar $k\to\infty$ em \eqref{3.3.1} e deduzir que $\Sigma_\infty$ acaba obedecendo \eqref{3.3.5}. Falta mostrar que \eqref{3.3.5} tem solução única. Suponha que $\Sigma^a_\infty$ e $\Sigma^b_\infty$ são duas soluções de \eqref{3.3.5}. Então $\Delta:=\Sigma^a_\infty-\Sigma^b_\infty$ satisfaz
\[
	\Delta = A\Delta A^T.
\]
Substituições recursivas levam a
\[
	\Delta = A^k\Delta(A^k)^T.
\]
Tomando o limite $k\to\infty$ mostra que $\Delta=0$ e portanto $\Sigma^a_\infty=\Sigma^b_\infty$.
\end{Dem}

A Equação \eqref{3.3.5} é chamada \textbf{equação de Lyapunov a tempo discreto}.

\begin{Exercicio}\label{3.3.6}
Suponha que $A$ seja estável. Se $\cov(x_0)=\Sigma_\infty$, então a função de covariância de $\{x_k\}$ é estacionária, ou seja, $\Sigma_k=\Sigma_\infty$ e $\Sigma_{k+m,\;k}=A^m\Sigma_\infty$ para todo $k>0$. Em geral, $\{x_k\}$ pode ser decomposta como
\[
	x_k=x_k'+x_k''
\]
onde a função de covariância de $\{x_k'\}$ é estacionária, e $\E\|x_k''\|^2\to0$ conforme $k\to\infty$. (A decomposição pode não ser única.)
\end{Exercicio}

\begin{Obs}\label{3.3.7}
É possível que $A$ não seja estável e ainda assim $\Sigma_k$ convergir. Um exemplo trivial é quando $\Sigma_0=0=Q$, caso em que $\Sigma_k=0$ independentemente de $A$. Também pode acontecer de $\Sigma_\infty$ não ser estritamente positiva definida. Outro exemplo trivial é quando $A=0$ e a matriz $GQG^T$ não tem posto completo. Esses dois exemplos sugerem que se a perturbação de entrada $\{w_k\}$ afetar todas as componentes do estado, então a estabilidade de $A$ pode ser necessária para a convergência de $\Sigma_k$, e a covariância limite $\Sigma_\infty$ será positiva definida. Essa intuição acaba sendo bem fundamentada, desde que a noção de afetar todos os componentes do estado seja adequadamente definida.
\end{Obs}
Um par de matrizes $(A,\;S)$ de dimensões $n\times n$ e $n\times g$, respectivamente, é dito ser \textbf{alcançável} se a matriz $n\times ng$ $[S\;\;AS\;\;\dots\;\;A^{n-1}S]$ tem posto $n$. Essa nomenclatura é justificada pelo exercício que segue.
\begin{Exercicio}\label{3.3.8}
Mostre a equivalência das afirmações:\\
1) $(A,\;S)$ é um par alcançável.\\
2) A matriz $\sum_{j=0}^{n-1}A^jSS^T(A^j)^T$, que é $n\times n$, é positiva definida.\\
3) Para todo $x\in\RR^n$ existe uma sequência $\{w_0,\;\dots,\;w_{n-1}\}$ de $\RR^g$ que leva o estado do sistema linear determinístico
\[
	x_{k+1}=Ax_k+Sw_k,\qquad k \ge0,
\]
do estado $x_0=0$ ao estado $x_n=x$.
\end{Exercicio}
\begin{Teo}\label{3.3.9}
Suponha que $(A,\;S)$ seja alcançável. Então as seguintes afirmações equivalem:\\
1) $A$ é estável.\\
2) A equação
\begin{equation}\label{3.3.10}
	\Sigma = A\Sigma A^T+ SS^T
\end{equation}
tem uma solução $\Sigma$ positiva definida.
\end{Teo}
\begin{Dem}
Se $A$ é estável, então pelo Teorema \ref{3.3.4} existe uma única solução $\Sigma$ em que
\[
	\Sigma = \sum_{j=0}^\infty
	 A^jSS^T(A^j)^T.
\]
Como $(A,\;S)$ é alcançável, $\Sigma$ é positiva definida por (\ref{3.3.8}--(2)).

Agora, suponha que $\Sigma$ é uma solução positiva definida de \eqref{3.3.10}. Então
\begin{equation}\label{3.3.11}
	\Sigma = A^k \Sigma (A^k)^T +
	\sum_{j=0}^{k-1}A^jSS^T(A^j)^T, \quad
	k\ge0.
\end{equation}
Seja $\lambda$ um autovalor de $A$ e $x\neq0$ um vetor tal que $A^Tx=\lambda x$. Multiplique \eqref{3.3.11} por $x^*$ à esquerda (transposto conjugado de $x$) e por $x$ à direita, e sendo $k=n-1$,
\[
	x^*\Sigma x=|\lambda|^{2(n-1)} x^*
	 \Sigma x+
	 x^*\left(\sum_{j=0}^{n-1}
	 A^jSS^T(A^j)^T\right)x.
\]
A matriz entre parênteses (...) da equação anterior é positiva definida por (\ref{3.3.8}--(2)) e então o último termo é estritamente positivo. Consequentemente $|\lambda|<1$  e então $A$ é estável.
\end{Dem}

Voltemos à equação da covariância em \eqref{3.3.1}. Seja $S$ alguma matriz tal que
\begin{equation}\label{3.3.12}
	GQG^T=SS^T.
\end{equation}
Tal matriz $S$ é chamada de \textbf{raiz quadrada} da matriz $GQG^T$.

\begin{Teo}\label{3.3.13}
Suponha que $(A,\;S)$ seja alcançável. Então as seguintes afirmações equivalem.\\
1) $A$ é estável.\\
2) $\Sigma_k$ converge a uma matriz positiva definida $\Sigma_\infty$ conforme $k\to\infty$.
\end{Teo}
\begin{Dem}
Segue imediatamente do Teorema \ref{3.3.9}.
\end{Dem}
\begin{Obs}
A raiz quadrada em \eqref{3.3.12} não é única. Entretanto, de {\rm(\ref{3.3.8}--(2))}, vemos que $(A,\;S)$ é alcançável se, e somente se $\sum_{j=0}^{n-1}A^jSS^T(A^j)^T=\sum_{j=0}^{n-1}A^jGQG^T(A^j)^T$ é positiva definida. Portanto, a alcançabilidade de $(A,\;S)$ não depende da escolha da raiz quadrada em \eqref{3.3.12}.
\end{Obs}





%%%%%%%%%%%%%% s
\section{Processos Gauss-Markovianos}
%%%%%%%%%%%%%% s
Um processo estocástico $\{x_k,\;k\ge0\}$ com valores em $\RR^n$ é Gauss-Markoviano (GM)\nomenclature{GM}{Gauss-Markoviano} se ele é: (a) Gaussiano, isto é, para todo $k$, a PD das variáveis aleatórias $x_0,\;\dots,\;x_k$ é Gaussiana, e (b) Markoviano, isto é,
\begin{equation}\label{3.4.1}
	p(x_{k+1}\;|\;x_k,\;\dots,\;x_0)
	= p(x_{k+1}\;|\;x_k).
\end{equation}

Temos visto que o estado do processo $\{x_k\}$ dado por
\[
    x_{k+1}=A_kx_k+w_k,
\]
em que $x_0,\;w_0,\;w_1,\;\dots$ são independentes e Gaussianos é GM.
Há a inversa desse fato, que deixamos como exercício.

Seja $\{x_k\}$ um processo GM. Suponha $\E x_0=0$, e considere
$\Sigma_k:=\cov(x_k)$, $\Sigma_{k+1,\;k}:=\cov(x_{k+1},x_k)$.
\begin{Exercicio}\label{3.4.2}
Mostre que
\begin{align}\label{3.4.3}
    \widehat x_{k+1\;|\;k} &:=
    \E(x_{k+1}\;|\;x_k,\dots,x_0)
    =\E(x_{k+1}\;|\;x_k)
    \nonumber\\
    &\phantom{:}=\Sigma_{k+1,\;k}\Sigma_k^{-1}x_k.
\end{align}
Agora, seja
\[
    w_k:=x_{k+1}-\widehat x_{k+1\;|\;k}.
\]
Mostre que as variáveis $x_0,\;w_0,\;w_1,\;\dots$ são independentes
e Gaussianas. Assim, o processo GM $\{x_k\}$ pode ser representado por
\begin{equation}\label{3.4.4}
    x_{k+1}=A_kx_k+w_k,
\end{equation}
com $A_k=\Sigma_{k+1,\;k}\Sigma_k^{-1}$.

\noindent
[Dica: a primeira igualdade em \eqref{3.4.3} segue da propriedade de Markov
\eqref{3.4.1}. Para a segunda igualdade e o restante do exercício, proceda
como segue. Mostre que $x_{k+1}-\Sigma_{k+1,\;k}\Sigma_k^{-1}x_k$ e $x_k$
não são correlacionadas calculando sua covariância.
Lembre-se de que duas variáveis aleatórias Gaussianas conjuntamente
são independentes se não estiverem correlacionadas. Por conseguinte,
$\E(\,\widehat x_{k+1}-\Sigma_{k+1,\;k}\Sigma_k^{-1}x_k\;|\;x_k)=0$
e então a segunda igualdade em \eqref{3.4.3} é verificada.]
\end{Exercicio}

No exercício, $w_k=x_{k+1}-\widehat x_{k+1\;|\;k}$ e $x_k$ juntos determinam
$x_{k+1}$. O último é conhecido no tempo $k$; no entanto o primeiro é independente
do passado, e portanto, desconhecido no tempo $k$. Por esse motivo
$w_k:=\widehat x_{k+1\;|\;k}-x_{k+1}$
é chamada de \textbf{a inovação}, ou nova informação sobre $x_{k+1}$
que não estava presente no passado. A representação \eqref{3.4.4} é chamada
\textit{representação da inovação} do processo $\{x_k\}$






%%%%%%%%%%%%%% s
\section{Notas}
%%%%%%%%%%%%%% s
Propriedades de sistemas lineares estocásticos são extensivamente discutidas
em \cite{anderson-moore1979}. Este livro também estuda alcançabilidade
e conceitos relacionados, incluindo controlabilidade e observabilidade.
Outras propriedades de sistemas lineares estocásticos são apresentadas
no Capítulo 7.




