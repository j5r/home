\documentclass[12pt,a4paper]{article}
\usepackage[T1]{fontenc}
\usepackage[utf8]{inputenc}
\usepackage[portuguese]{babel}
\usepackage[dvipsnames]{xcolor}
\usepackage[left=2cm,right=2cm,top=2cm,bottom=2cm]{geometry}
\usepackage{amsfonts,amssymb,amsmath}
\setlength{\parindent}{0pt}
\newenvironment{ans}{\color{blue}\begin{quote}}{\end{quote}}

\begin{document}
5) A resistência de um certo tipo de cabo de aço é uma variável aleatória modelada pela distribuição Normal com desvio padrão igual a 6 kgf. Uma amostra de tamanho 25 desses cabos, escolhida ao acaso, forneceu média igual a 9,8 kgf. Teste as hipóteses $\mu = 13$ versus $\mu = 8$ e tire suas conclusões a um nível de significância de $10\%$.



\dotfill

$\bigstar$ 
\[
x \sim N(\mu,6^2)
\]
\[
\bar X_{(n=25)} \sim N(\mu, 36/25)
\]

$\bigstar$ Significância $\alpha=0.1$

\qquad$H_0: \qquad \mu=13$

\qquad$H_a: \qquad \mu=8$

$\bigstar$ Vamos determinar a região crítica.

Tomemos $Z\sim N(0,1)$.

\[\begin{aligned}
0.1&=P(Z<-1.2815) \\
&= P\left(\frac{\sqrt{25}(\bar X - \mu)}{\sqrt{36/25}} <-1.2815\;\Big|\;\mu=13\right) 
\\
&= P\left(\frac{25(\bar X - \mu)}{6} < -1.2815\;\Big|\;\mu=13\right)
\\
&= P(\bar X <\mu-1.2815\cdot 6/25\;|\;\mu=13)
\\
&= P(\bar X <13-1.2815\cdot 6/25)
\\
&= P(\bar X<12.6924)
\end{aligned}\]

Assim, obtivemos $RegiaoCritica=\{x; x<12.6924\}$.\\

$\bigstar$ Teste: 
\begin{quote}
Sob o nível de significância $\alpha=0.1$,\\
se $\bar x_0 \in RegiaoCritica$, rejeitamos $H_0$;\\
se $\bar x_0\not \in RegiaoCritica$, aceitamos $H_0$.
\end{quote}

Evidência observada: $\bar x_0=9.8$

$\bigstar$ Decisão: como a evidência observada pertence à região crítica, decidimos rejeitar $H_0$ ao nível de significância $\alpha=0.1$.


Nível Descritivo ou Valor P:
\[
P=P(X\text{ é um valor mais extremo que a evidência observada}\;|\;H_0\ verdadeira)
\]
\[
P\le\alpha \Rightarrow \mbox{rejeita-se $H_0$}.
\]

Neste exercício,
\[
P=P(X< \bar x_0\;|\;H_0\ verdadeira)=P(X< 9.8\;|\;\mu=13)=0.00383
\]

Nível descritivo $P=0.00383\le \alpha=0.1$ (implica a rejeição de $H_0$).


\end{document}