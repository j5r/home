\documentclass[12pt,a4paper]{article}
\usepackage[T1]{fontenc}
\usepackage[utf8]{inputenc}
\usepackage[left=2cm,right=2cm,top=2cm,bottom=2cm]{geometry}
\usepackage{amsfonts,amssymb,amsmath,hyperref,color}
\setlength{\parskip}{12pt}


\begin{document}
\noindent Abreviações:
\begin{description}
\item[JS] JavaScript (linguagem de programação EcmaScript, conhecida como JavaScript)
\item[lh:] localhost
\end{description}
Um objeto JS é uma estrutura de dados tal como um dicionário.

\section{Rotas}
São endereços na URL em que um fluxo de informações fica
disponibilizado pelo servidor. Uma rota começa por um
endereço de um host, adicionada de uma porta para comunicação
e termina em  algum sufixo. 

Dada a definição da rota, apenas a porta e o sufixo são 
mutáveis, já que o host (servidor) não se altera. Fixando-se uma
porta, podemos exemplificar algumas rotas:
\begin{itemize}
\item ``/''
\item ``/user''
\item ``/user/channel''
\item ``/clients?filter=idade\&type=gt\&value=18''
\end{itemize}
Essa última é uma rota formatada como uma Query, indicando o objeto JS

\verb|{filter: "idade", type: "gt", value: "18"}|

A rota completa do último exemplo para a porta 3030 é então (lh será o domínio do site quando este for ao ar).

lh:3030/clients?filter=idade\&type=gt\&value=18

Por abuso de linguagem/notação, às vezes chamamos o sufixo de rota.



\section{Métodos HTTP}
Há outros métodos, mas os principais são descritos.
\begin{description}
\item[GET] pegar informação

\item[POST] criar informação

\item[PUT] alterar informação

\item[DELETE] apagar informação
\end{description}

\section{Códigos de estado HTTP}
São códigos indicando o funcionamento dos métodos HTTP, quando sucedem ou quando dão erros. Um código é um número inteiro, mundialmente padronizado, indicando a causa do erro. Para mais detalhes, clique aqui \href{https://pt.wikipedia.org/wiki/Lista_de_c%C3%B3digos_de_estado_HTTP}{\color{blue}X}.



\section{Estruturas de uma requisição HTTP}
Uma requisição é o pedido de informações para um servidor.
É constituída de um objeto JS, contendo os seguintes campos
descritos abaixo. A estrutura para certa finalidade pode ser qualquer 
uma a priori, mas por certas razões, se escolhe uma ou outra.

\begin{description}
\item[QUERY] Permite o fluxo de informações diretamente na URL a partir da rota. É um objeto JS parseado na URL. Inicia-se com ``?''; os campos são informados como ``campo=valor'', e separa-se os campos por ``\&''. Caracteres não imprimíveis ou que não sejam ASCII são devidamente codificados. É interessante para filtrar dados em uma busca.

\item[BODY] É um objeto JS. Útil para o fluxo de várias informações. Precisa ser codificado/decodificado de acordo com algum formato, como JSON, XML, ou outros. 

\item[PARAMS] Permite o fluxo de apenas uma informação. São parâmetros adicionados diretamente nas rotas: \verb|lh:port/route/:id| para receber um parâmetro \verb|id| diretamente pela rota. O cliente envia, por exemplo \verb|lh:port/route/a3b8| em que \verb|a3b8| é a \verb|id| enviada.

\item[HEADERS] É um objeto JS, que permite o fluxo de várias informações. Pode servir para enviar uma id autorizando a execução de algo
(envie a id do usuário para fazer login, por exemplo)
Devolve ao cliente os, tais como o tão conhecido \emph{404 ``Page not found''}.
\end{description}



\end{document}