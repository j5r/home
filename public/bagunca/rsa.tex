\documentclass[a4paper, 12pt]{article} %tamanho da fonte, tipo de papel

% espaçamentos das margens da folha de papel
\usepackage[left=2cm, right=2cm, top=2cm, bottom=2cm]{geometry}

% font-encoding: para que possibilite copiar/colar do PDF sem erros
\usepackage[T1]{fontenc}

% input-encoding: para que possibilite caracteres acentuados diretamente do teclado
\usepackage[utf8]{inputenc}

% um pacote de cores não padrão; peguei a paleta de cores "dvipsnames"
\usepackage[dvipsnames]{xcolor}

% para inserção de equações matemáticas corretamente.
\usepackage{amsmath, amssymb, amsfonts}

% incluir tipografia (fonte) Times New Roman
\usepackage{txfonts}

% para incluir urls/links
\usepackage{hyperref} %configurando o hyperref
\hypersetup{
    colorlinks=true,
    linkcolor=blue,
    filecolor=magenta,      
    urlcolor=ForestGreen,
}

% espaçamento entre parágrafos
\setlength{\parskip}{12pt}

% e início de parágrafo
\setlength{\parindent}{12pt}

% tradução de algumas palavras-chave para o português
\usepackage[portuguese]{babel}

\title{Criptografia RSA}
\author{Junior R. Ribeiro}

\begin{document}

\maketitle


\section{Criptografia RSA}
A Criptografia RSA (Rivest-Shamir-Adleman são os autores deste modelo de criptografia) consiste em um modelo matemático de criptografia assimétrica, isto é, o processo de decodificação é diferente do processo de codificação. Ela é fortemente dependente da \textbf{dificuldade de fatoração de números compostos muito grandes}; é isso que lhe garante segurança nos dados.

Os número primos são aqueles que são divisíveis somente por $1$ e por ele mesmo, tais como $2,\; 3,\; 5,\; 7,\; 11,\; 13,\; 17,\; 19,\; 23,\; 29,\; \dots$, e os demais números são compostos porque são produtos de números primos, como exemplo $6=2\cdot 3,\; 15=3\cdot 5$, etc.

Fatorar um número composto com milhares de dígitos $C$ pode ser uma tarefa simples, se este número for composto pelo produto de dois números primos $P$ e $Q$, em que $P$ seja enorme e $Q$ seja pequeno, digamos, $23$. Basta fazermos algumas tentativas pelos números primos listados acima, até que encontremos $23$ e conseguimos um fator primo para $C$. Agora, se $P$ e $Q$ forem gitantescos, a tarefa de fatoração é impraticável mesmo com os melhores supercomputadores existentes e a existir, pois precisamos testar cada um dos possíveis números primos anteriores para saber se são um fator para $C$. É aí que reside a segurança da RSA\footnote{Para mais detalhes, veja, por exemplo, \href{https://pt.wikipedia.org/wiki/RSA_(sistema_criptográfico)}{RSA (sistema criptográfico)}.}. 
% nota de rodapé \footnote
% inserindo link \href
% não insira nenhum espaço no endereço, pois não funcionará.
% para inserir link para um e-mail, use como endereço "mailto:joao@hotmail.com"

\section{Modelo}
O RSA consiste de duas chaves, a pública e a privada, que são dois números grandes. A chave pública pode ser conhecida por qualquer pessoa, inclusive os hackers. Ela é o produto de dois primos grandes. Para ``hackear'' o RSA, o hacker precisará fatorar esse produto e descobrir quem são os dois primos que o compõem, o que, como já dito, é impraticável com a tecnologia atual. A chave privada deve ser mantida em segredo pelo responsável pela segurança do sistema. A mensagem é codificada utilizando apenas a chave pública, já a decodificação é feita usando a chave pública e a privada; esta última é calculada a partir dos números primos que geram a chave pública. Por isso é importante criar o par de chaves e destruir os números primos, para que ninguém mais possa quebrar a criptografia; e logicamente guardar a chave privada em segredo.

O modelo a seguir é um modelo particular do RSA, que é mais genérico.

{
\color{ForestGreen} \textbf{Nota:} Representamos o resto $r$ da divisão de $a$ por $b$ indicando que $a$ é igual a $r$ módulo $b$, ou 
\[
  a = r \pmod{b}.
\]
}

Considere dois números primos grandes $P$ e $Q$ escolhidos de modo que deixem resto $5$ quando divididos por $6$, ou seja, $P=5\pmod{6}$ e $Q=5\pmod{6}$.

A chave pública é
\begin{equation} \label{eq:chavePublica}
\alpha = P \cdot Q.
\end{equation}

A chave privada é 
\begin{equation} % equações numeradas
  \label{eq:chavePrivada} % link para referência cruzada 
\beta = 4 k - 1, 
\end{equation}
em que $k$ é calculado por
\[
 k = \frac{(P-1)(Q-1) + 2}{6} %fração  
\]


A codificação de uma mensagem $M$ é feita tomando o resto da divisão de $M^3$ pela chave pública $\alpha$, vamos chamar esse resto de $M_r$. Escrito de outra maneira, temos
\[
M^3 = M_r\pmod{\alpha}.
\]
A mensagem codificada é $M_r$.


A decodificação é feita tomando o resto da divisão de $M_r^{\beta}$ pela chave pública $\alpha$. Ou matematicamente,
\[
M_r^\beta = M \pmod{\alpha}.
\]




\subsection{Exemplo numérico}
Tomemos $P=11$ e $Q=41$, pois são $11 = 5 \pmod{6}$ e $41 = 5 \pmod{6}$.

A chave pública é 
\[ % equações não numeradas
  \alpha = 11\cdot 41 = 451,
\]

Calculamos $k$,
\[
  k = \frac{(11-1)(41-1) + 2}{6} = \frac{10\cdot 40 + 2}{6} = 67,
\]
e obtemos a chave privada
\[
\beta = 4\cdot 67 - 1 = 267.  
\]

Nossa mensagem não pode ser maior que a chave pública $\alpha$, para que haja possibilidade de recuperar a mensagem após a codificação e decodificação.

Suponha que nossa mensagem seja $M = 96$. Vamos calcular $M_r$
\[
 M^3 = 96^3 = 96^2 \cdot 96 = {\color{red}9216} \cdot 96 = {\color{red}196} \cdot 96 = 18816 = 325\pmod{451}
\]
e assim, $M_r=325$ é a mensagem \textbf{\color{ForestGreen}codificada}. Perceba em vermelho, que trocamos $96^2$ pelo seu resto que é $196$ módulo $451$. Este é o procedimento quando trabalhamos com a aritmética modular, assunto para você pesquisar.

Agora, vamos decodificar a mensagem $M_r=325$ usando ambas as chaves.

\[
M_r^{267} = 325^{267}  \pmod{451}
\]
Vejamos como fica $M_r^2$
\[
M_r^{2}= 325^2 = 105625  = 91 \pmod{451}.
\]


Vamos olhar com cuidado o expoente
\[
267 = \underbrace{2+2+...+2}_{133 \mbox{ vezes}} + 1. 
\]
com isso, podemos substituir no problema original esse resultado (apliação de aritmética modular)
\[
M_r^{267} = 325^{267} = \underbrace{91 \times 91 \times ... \times 91}_{133 \mbox{ vezes}}\times 325 = 325\cdot 91^{133}\pmod{451}.
\]

Veja que o problema era o expoente $267$, agora o problema é menor, $133$. Vamos fazer essa redução repetidas vezes. Vamos chamar essa base $b_1 = M_r^2 = 91 \pmod{451}$.

\dotfill\,

Vejamos como fica $b_1^2 \pmod{451}$.
\[
b_1^2 = 91^2 = 8281 = 163 \pmod{451}.
\]
Vamos olhar de novo o novo expoente
\[
133 = \underbrace{2+2+...+2}_{66 \mbox{ vezes}} + 1.
\]
Com isso, podemos substituir no problema original
\[
\begin{aligned}
M_r^{267} & = 325^{267} = 325\cdot 91^{133} = 325\cdot (\underbrace{163 \times 163 \times ... \times 163}_{66 \mbox{ vezes}}\times 91) \\
          & = 325\cdot 91\cdot 163^{66}=29575\cdot 163^{66}=260\cdot 163^{66}\pmod{451}.
\end{aligned}
\]

Veja que o problema era o expoente $133$, agora o problema é menor, $66$. Vamos chamar essa base $b_2 = M_r^4 = 163 \pmod{451}$.

\dotfill\,

Vejamos como fica $b_2^2 \pmod{451}$.
\[
b_2^2 = 163^2 = 26569 = 411 \pmod{451}.
\]

Vamos olhar de novo o novo expoente
\[
66 = \underbrace{2+2+...+2}_{33 \mbox{ vezes}}.
\]

Com isso, podemos substituir no problema original
\[
M_r^{267}  = 325^{267} = 260\cdot (\underbrace{411 \times 411 \times ... \times 411}_{33 \mbox{ vezes}}) = 260\cdot 411^{33}\pmod{451}.
\]

Veja que o problema era o expoente $66$, agora o problema é menor, $33$. Vamos chamar essa base $b_3 = M_r^8 = 411 \pmod{451}$.


\dotfill\,

Vejamos como fica $b_3^2 \pmod{451}$.
\[
b_3^2 = 411^2 = 168921 = 247 \pmod{451}.
\]

Vamos olhar de novo o novo expoente
\[
33 = \underbrace{2+2+...+2}_{16 \mbox{ vezes}}+1.
\]

Com isso, podemos substituir no problema original
\[
  \begin{aligned}
M_r^{267} &= 325^{267} = 260\cdot (\underbrace{247 \times 247 \times ... \times 247}_{16 \mbox{ vezes}}\times 411) \\
& = 260\cdot 411\cdot 247^{16} = 106860\cdot 247^{16}=424\cdot 247^{16}\pmod{451}.
\end{aligned}
\]

Veja que o problema era o expoente $33$, agora o problema é menor, $16$. Vamos chamar essa base $b_4 = M_r^{16} = 247 \pmod{451}$.


\dotfill\,

Vejamos como fica $b_4^2 \pmod{451}$.
\[
b_4^2 = 247^2 = 61009 = 124 \pmod{451}.
\]

Vamos olhar de novo o novo expoente
\[
16 = \underbrace{2+2+...+2}_{8 \mbox{ vezes}}.
\]

Com isso, podemos substituir no problema original
\[
M_r^{267} = 325^{267} = 424\cdot (\underbrace{124 \times 124 \times ... \times 124}_{8 \mbox{ vezes}}) = 424\cdot 124^{8}\pmod{451}.
\]

Veja que o problema era o expoente $16$, agora o problema é menor, $8$. Vamos chamar essa base $b_5 = M_r^{32} = 124 \pmod{451}$.

\dotfill\,

Vejamos como fica $b_5^2 \pmod{451}$.
\[
b_5^2 = 124^2 = 15376 = 42 \pmod{451}.
\]

Vamos olhar de novo o novo expoente
\[
8 = 2+2+2+2.
\]

Com isso, podemos substituir no problema original
\[
M_r^{267} = 325^{267} = 424\cdot (42\times 42\times 42\times 42) = 424\cdot 42^{4}\pmod{451}.
\]

Veja que o problema era o expoente $8$, agora o problema é menor, $4$. Vamos chamar essa base $b_6 = M_r^{64} = 42 \pmod{451}$.

\dotfill\,

Vejamos como fica $b_6^2 \pmod{451}$.
\[
b_6^2 = 42^2 = 1764 = 411 \pmod{451}.
\]

Vamos olhar de novo o novo expoente
\[
4 = 2+2.
\]

Com isso, podemos substituir no problema original
\[
M_r^{267} = 325^{267} = 424\cdot 411\cdot 411 = 71622504 = 96\pmod{451}.
\]

E obtemos a mensagem original.

\subsection{Comentário}
O processo de criptografia foi rápido, pois escolhemos um modelo especificamente para essa finalidade. Veja que para codificar, precisamos apenas elevar a mensagem $M$ à terceira potência. O valor de $k$ e de $\beta$ foi determinado especificamente para esse número três, bem como a exigência de que os números primos sejam iguais a $5$ módulo $6$.

Já para decodificar, o processo se tornou bem mais árduo porque a chave privada acaba sendo um número grande, e por isso precisamos fazer reduções sucessivas até obtermos a mensagem descriptografada.
\end{document}